\documentclass{article}

\usepackage{amsmath}
\usepackage{amssymb}
\usepackage{mathrsfs}
\usepackage{hyperref}
\usepackage{mathtools}
\newtheorem{thm}{Theorem}
\newtheorem{defn}{Definition}
\newtheorem{prop}{Proposition}
\newtheorem{rmk}{Remark}
\newtheorem{cor}{Corollary}
\newtheorem{lem}{Lemma}

\title{Summary of Curvatures of Left Invariant Metrics on Lie Groups}
\author{Samuel Wallace}

\begin{document}

\maketitle

\indent This is a summary of the paper "Curvatures of Left Invariant metrics on Lie Groups" by John Milnor available \href{https://core.ac.uk/download/pdf/82428733.pdf}{here}.


\section{Sectional Curvature}

\indent Let $G$ be an $n$-dimensional Lie group, and $\mathfrak{g}$ its associated Lie algebra. Choosing some basis $e_1, \ldots, e_n$ for $\mathfrak{g}$, there is obviously only one metric making this basis orthonormal. In fact, we can choose exactly one metric making the inner product $\langle e_i, e_j \rangle$ is the $i-j$-th component of a specific matrix. So there are a $\frac{1}{2}n(n+1)$ dimensional manifold of left-invariant metrics on $G$.
\indent The sectional curvature of a metric is defined to be 
\begin{equation}\label{eq:1}
    \kappa(x,y) = \langle R_{xy}(x),y \rangle
\end{equation}
\indent For orthonormal vectors $x,y$. This is Gaussian curvature of the surface swept out by the vectors $x,y$. \\
\indent The structure constant of a Lie group are the numbers $\alpha_{ijk}$ such that
\begin{equation}\label{eq:2}
    [e_i,e_j]=\sum_k \alpha_{ijk}e_k
\end{equation}
The next fact is not practically useful, but theoretically interesting.
\begin{lem}

The sectional curvature is given in terms of structure constants by 
\begin{align*}\label{eq:3}
    \kappa(e_i,e_j) = \sum_k  &\frac{1}{2}\alpha_{ijk}(-\alpha_{ijk}+\alpha_{jki} + \alpha_{kij}) - \\ &\quad \frac{1}{4}(\alpha_{ijk}-\alpha_{jki}+\alpha_{kij})(\alpha_{ijk}+\alpha_{jki}+\alpha_{kij}) - \alpha_{kii}\alpha_{kjj} 
\end{align*}
\end{lem}
\indent The next fact is slightly more useful.

\begin{lem}\label{lem:1}
If $\mathrm{ad}(u)$ is skew-adjoint, then $\kappa(u,v) \geq 0$ when $u \perp [v,\mathfrak{g}]$
\end{lem}


There is an important corollary:

\begin{cor}
If $u$ belongs to the center of $\mathfrak{g}$ (i.e. $[v,\mathfrak{g}]=0$), then for any left-invariant metric and any vector $v$, $\kappa(u,v) \geq 0$.
\end{cor}

\begin{lem}
A left-invariant metric on a connected Lie group is also right-invariant iff $\mathrm{ad}(x)$ is skew-adjoint for all $x \in \mathfrak{g}$. A Lie group admits a bi-invariant metric iff it is isomorphic to a Cartesian product of a compact group and a commutative group.
\end{lem}

\begin{cor}
Every compact Lie group admits a left-invariant and a bi-invariant metric so that $K \geq 0$ for all sectional curvatures.
\end{cor}

\begin{thm}
A Lie Group with left-invariant metric is flat iff the associated Lie algebra splits as an orthogonal direct sum $\mathfrak{b}\oplus \mathfrak{u}$ where $\mathfrak{b}$ is a commutative subalgebra, $\mathfrak{u}$ is a commutative ideal, and if $\mathrm{ad}(b)$ is skew-adjoint for every $b\in \mathfrak{b}$.
\end{thm}

\indent The necessary and sufficient conditions for a left-invariant metric to have negative sectional curvature is that $\mathfrak{g}=[\mathfrak{g},\mathfrak{g}]+\mathbb{R}x$ and that $\mathrm{ad}(x)\restriction_{[\mathfrak{g},\mathfrak{g}]}$ has eigenvalues with positive real part. $K \leq 0$ groups have been classified in the following statements.

\begin{thm}
If a connected Lie group $G$ has a left-invariant metric with $K \leq 0$, then it is solvable. If a left-invariant Haar measure is also right-invariant (unimodular), then the $K=0$.
\end{thm}

\section{Ricci Curvature}
\indent Another curvature is given by the Ricci curvature, defined by 
\begin{equation}
    r(x)=\sum_i \kappa(x,e_i) = \sum_i \langle R_{x e_i}(x),e_i \rangle
\end{equation}
\indent For a unit vector $u$, $r(u)$ is the Ricci curvature of the direction. It is equal to $(n-1)$ times the average of the sectional curvature of all tangent planes containing $u$. It will become more convenient to work with the \textit{Ricci transformation}, defined by
\begin{equation}
    \widehat{r}(x)=\sum_i R_{e_i x}(e_i)
\end{equation}
\indent Which gives the relation
\begin{equation}
    r(x)=\langle \widehat{r}(x),x \rangle
\end{equation}

The eigenvalues of $\widehat{r}$ are called the principal Ricci curvatures. Now back to left-invariant metrics. 
\begin{lem}
If $\mathrm{ad}(u)$ is skew-ajoint, then $r(u) \geq 0$, where there is only equality if $u \perp [\mathfrak{g},\mathfrak{g}]$. 
\end{lem}

\begin{thm}
A connected Lie group admits a left-invariant metric with all Ricci curvatures strictly positive iff it is compact with finite funamental group.
\end{thm}

\begin{lem}
If $u \perp [\mathfrak{g},\mathfrak{g}]$, then $r(u) \leq 0$ with equality iff $\mathrm{ad}(u)$ is skew-adjoint.
\end{lem}

\begin{defn}

A Lie algebra is \textbf{nilpotent} if some term in the series
\begin{equation}
    \mathfrak{g} \supset [\mathfrak{g}, \mathfrak{g}] \supset [\mathfrak{g},[\mathfrak{g},\mathfrak{g}]] \supset \ldots
\end{equation}
is zero.
\end{defn}

\begin{thm}
Suppose $\mathfrak{g}$ is nilpotent but not commutative. Then for any left-invariant metric there is a direction of strictly negative Ricci curvature and one of strictly positive Ricci curvature.
\end{thm}

\begin{thm}

If the Lie algebra of $G$ contains linearly independent vectors $x,y,z$ so that $[x,y]=z$, then there is a left-invariant metric so that $r(x) < 0$ and $r(z) > 0$.
\end{thm}

\section{Scalar Curvature}

\begin{defn}

Choose an orthonormal basis $e_i$ for the tangent space, then 
\begin{equation}
    \rho = \sum_i r(e_i)
\end{equation} is the scalar curvature. It is $n(n-1)$ times the average of all sectional curvatures at a point.
\end{defn}

\begin{thm}
If $G$ is solvable, then every left-invariant metric on $G$ is either flat, or has strictly negative curvature.
\end{thm}

\begin{cor}
If $G$ is solvable and unimodular, then every left-invariant metric on $G$ is either flat, or has both positive and negative sectional curvatures.
\end{cor}

\begin{thm}
If $\mathfrak{g}$ is noncommutative, then $G$ has a left-invariant metric of strictly negative curvature.
\end{thm}

\begin{thm}[Wallach]
If the universal covering of $G$ is not homeomorphic to Euclidean space, then $G$ admits a left-invariant metric of strictly positive scalar curvature.
\end{thm}

\end{document}