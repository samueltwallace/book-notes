\documentclass{../booknotes}

\usepackage{import}

\booktitle{Functional Analysis, Sobolev Spaces, and PDEs}
\bookauthor{Haim Brezis}
\notesauthor{Samuel T. Wallace}

\begin{document}

\maketitle
\begin{pubdescrip}

\indent Uniquely, this book presents a coherent, concise and unified way of combining elements from two distinct “worlds,” functional analysis (FA) and partial differential equations (PDEs), and is intended for students who have a good background in real analysis. This text presents a smooth transition from FA to PDEs by analyzing in great detail the simple case of one-dimensional PDEs (i.e., ODEs), a more manageable approach for the beginner. \\
\indent Although there are many books on functional analysis and many on PDEs, this is the first to cover both of these closely connected topics. Moreover, the wealth of exercises and additional material presented, leads the reader to the frontier of research. This book has its roots in a celebrated course taught by the author for many years and is a completely revised, updated, and expanded English edition of the important “Analyse Fonctionnelle” (1983). Since the French book was first published, it has been translated into Spanish, Italian, Japanese, Korean, Romanian, Greek and Chinese. The English version is a welcome addition to this list. The first part of the text deals with abstract results in FA and operator theory. \\
\indent The second part is concerned with the study of spaces of functions (of one or more real variables) having specific differentiability properties, e.g., the celebrated Sobolev spaces, which lie at the heart of the modern theory of PDEs. The Sobolev spaces occur in a wide range of questions, both in pure and applied mathematics, appearing in linear and nonlinear PDEs which arise, for example, in differential geometry, harmonic analysis, engineering, mechanics, physics etc. and belong in the toolbox of any graduate student studying analysis.
\end{pubdescrip}
\begin{transcribernote}
	\indent These notes were taken for self-study in Spring 2020. I was already familiar with some aspects of FA, through a mathematical methods course which essentially was a "FA for Scientists' class where distributions and the basic theory of operators on Banach and Hilbert spaces were covered, finishing with the spectral theorem.\\
	\indent The reader is assumed to have a semester in point-set topology, some exposure to PDEs, and familiarity with abstract vector spaces. These notes were taken without proof for brevity, and like all my notes, is meant as a `theorem cheat sheet,' rather than a comprehensive self-teaching course.
\end{transcribernote}

\tableofcontents


\import{./}{ch1.tex}
\import{./}{ch2.tex}
\import{./}{ch3.tex}
\import{./}{ch4.tex}

\end{document}
