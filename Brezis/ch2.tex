\section{The Uniform Boundedness Principle and the Closed Graph Theorem}

\subsection{The Baire Category Theorem}

\begin{thm}[Baire]
	Let $X$ be a complete metric space and $ \left( X_n \right)_{n \geq q}$ be a sequence of closed subsets in $X$. If
	\[
	\mathrm{Int} X_n = \emptyset
	\]
	Then
	\[
		\mathrm{Int} \left( \bigcup_{n}X_n \right) = \emptyset
	\]
	
\end{thm}

\subsection{The Uniform Boundedness Principle}

\begin{defn}
	Let $E$ and $F$ be two normed vector spaces. Let $ \mathcal{L}(E,F)$ be the space of continuous (bounded) \textit{linear} operators equipped with the norm
	\[
		\Vert T \Vert_{ \mathcal{L}(E,F)} = \sup_{ \Vert x \Vert \leq 1} \Vert Tx \Vert
	\]
	And we write $ \mathcal{L}(E) = \mathcal{L}(E,E)$.
\end{defn}

\begin{thm}[Banach-Steinhaus, uniform boundedness principle]
	Let $E$ and $F$ be two Banach spaces and let $ \left( T_i \right)_{i \in I}$ be a family of continuous linear operator from $E$ into $F$. If
	\[
	\forall x \in E \hspace{4pt} \sup_{i \in I} \Vert T_i x \Vert \leq \infty
	\]
	Then
	\[
		\sup_{i \in I} \Vert T_i \Vert_{ \mathcal{L}(E,F)}
	\]
	
\end{thm}

\begin{cor}
	Let $E$ and $F$ be two Banch spaces. Let $ \left( T_n \right)$ be a sequence of continuous linear operators from $E$ into $F$ such that $ \forall x \in E \hspace{4pt} T_n x$ converges (to a limit we call $Tx$). Then
\begin{enumerate}
	\item $\sup_n \Vert T_n \Vert_{ \mathcal{L}(E,F)} < \infty$
	\item $ T \in \mathcal{L}(E,F)$
	\item $ \Vert T \Vert_{ \mathcal{L}(E,F)} \leq \mathrm{liminf}_n \Vert T_n \Vert_{ \mathcal{L}(E,F)}$
\end{enumerate}	
\end{cor}

\begin{cor}
Let $G$ be a Banach space and let $B$ be a subset of $G$. If
\[
	\forall f \in G^* \hspace{4pt} f(B) \text{ is bounded in } \mathbb{R}
\]
Then $B$ is bounded.
\end{cor}

\begin{cor}
Let $G$ be a Banach space and let $B^*$ be a subset of $G^*$. If
\[
	\forall x \in G \hspace{4pt} \langle B^*,x \rangle \text{ is bounded in } \mathbb{R}
\]
Then $B^*$ is bounded.
\end{cor}


\subsection{The Open Mapping Theorem and the Closed Graph Theorem}

\begin{thm}[Open Mapping Theorem]
Let $E$ and $F$ be two Banach spaes and let $T$ be a continuous linear operator from $E$ into $F$ that is surjective. Then there exists $ \delta > 0$ such that
\[
	T \left( B_E(0,1) \right) \supset B_F(0, \delta)
\]
Which says $T$ is an open mapping.
\end{thm}

\begin{cor}
Let $E$ and $F$ be two Banach spaces and let $T$ be a continuous linear operator from $E$ into $F$ that is bijective. Then $T^{-1}$ is also continuous.
\end{cor}

\begin{cor}
Let $E$ be a vector space with two norms $ \Vert \cdot \Vert_1, \Vert \cdot \Vert_2$ with both make $E$ into a Banach Space, and that there is a constant $C \geq 0$ such that
\[
\Vert x \Vert_2 \leq C \Vert x \Vert_1
\]
Then the two norms are equivalent.
\end{cor}

\begin{thm}[Closed Graph Theorem]
	Let $E$ and $F$ be two Banach spaces. Let $T$ be a linear operator from $E$ to $F$. If the graph of $T$, $G(T)$, is closed in $ E \times F$, then $T$ is continuous.
\end{thm}

\subsection{Complementary Subspaces. Right and Left Invertibility of Linear Operators}

\begin{thm}
Let $E$ be a Banach space. Assume that $G$ and $L$ are two closed linear suspaces such that $G+L$ is closed. Then there exists a constant $C \geq 0$ such that $z \in G+L \Rightarrow z = x + y$ with $ C \Vert z \Vert \geq \Vert x \Vert, x \in G$ and $ C \Vert z \Vert \geq \Vert y \Vert, y \in L$.

\end{thm}

\begin{defn}
Let $G \subset E$ be a closed subspace of a Banach space $E$. A subspace $L \subset E$ is said to be a \textit{topological complement} or simply a \textit{complement} of $G$ if
\begin{enumerate}
	\item $L$ is closed
	\item $ G \cap L = 0$ and $G+L = E$
\end{enumerate}
We also say $G$ and $L$ are \textit{complementary} subspaces of $E$. If this holds, then every $z$ can be decomposed into components in $G$ and $L$, for which the projection operators are continuous.
\end{defn}

\begin{defn}
	Let $T \in \mathcal{L}(E,F)$. A \textit{right inverse} is an operator $S \in \mathcal{L}(F,E)$ such that $ T \circ S = I_F$. A \textit{left inverse} is an operator $S \in \mathcal{L}(F,E)$ such that $S \circ T = I_E$.
\end{defn}

\begin{thm}
	Let $T \in \mathcal{L}(E,F)$ be surjective. The following are equivalent:
	\begin{enumerate}
		\item $T$ admits a right inverse.
		\item $N(T) = T^{-1}(0)$ admits a complements in $E$.
	\end{enumerate}
\end{thm}

\begin{thm}
	Let $T \in \mathcal{L}(E,F)$ be injective. The following are equivalent:
	\begin{enumerate}
		\item $T$ admits a left inverse.
		\item $R(T) = T(E)$ is closed and admits a complement in $F$.
	\end{enumerate}
\end{thm}

\subsection{Orthogonality Revisited}

\begin{prop}
Let $G$ and $L$ be two closed subspaces in $E$. Then
\[
	G \cap L = \left( G^{\perp} + L^{\perp} \right)^{\perp}
\]
\[
	G^{\perp} \cap L^{\perp} = \left( G + L \right)^{\perp}
\]

\end{prop}


\begin{cor}
	\[
		\left( G \cap L \right)^{\perp} \supset \overline{G^{\perp} + L^{\perp}}
	\]
	\[
		\left( G^{\perp} \cap L^{\perp} \right)^{\perp} = \overline{G + L}
	\]
	
\end{cor}

\begin{thm}
Let $G$ and $L$ be two closed subpsaces in a Banach spaces $E$. The following are equivalent:
\begin{enumerate}
	\item $G+L$ is closed in $E$
	\item $G^{\perp} + L^{\perp}$ is closed in $E^{*}$
	\item $G+L = \left( G^{\perp} + L^{\perp} \right)^{\perp}$
	\item $G^{\perp} + L^{\perp} = \left( G \cap L \right)^{\perp}$
\end{enumerate}
\end{thm}

\subsection{An Introduction to Unbounded Linear Operators. Definition of the Adjoint}

\begin{defn}
	Let $E$ and $F$ be two Banach spaces. An \textit{unbounded linear operator} from $E$ into $F$ is a linear map $A: D(A) \subset E \to F$ where $D(A)$ is a linear subspace called the \textit{domain} of $A$. \\
	\indent $A$ is \textit{bounded} (or \textit{continuous}) if $D(A) = E$ and there is a $c \geq 0$ such that
	\[
	\Vert Au \Vert \leq c \Vert u \Vert
	\]
	The norm of a bounded operator is defined as
	\[
		\Vert A \Vert_{ \mathcal{L}(E,F)} = \sup_{u \neq 0} \frac{ \Vert Au \Vert}{ \Vert u \Vert} 
	\]
	Some additional definitions are as follows:
	\begin{enumerate}
		\item $G(A) = \left\{ (u, Au): u \in D(A) \right\} \subset E \times F$, the Graph of $A$
		\item $R(A) = \left\{ Au: u \in D(A) \right\} \subset F$, the range of $A$
		\item $N(A) = \left\{ u \in D(A): Au=0 \right\} \subset E$, the kernel of $A$.
	\end{enumerate}
	An operator $A$ is \textit{closed} if $G(A)$ is closed in $E \times F$.
	
\end{defn}

\begin{defn}
	Let $A: D(A) \subset E \to F$ be an unbounded linear operator that is \textit{densely defined} ($D(A)$ is dense in $E$). We introduce a new operator $A^{*}: D(A^{*}) \subset F^{*} \to E^{*}$ as follows. First we define
	\[
		D(A^{*}) = \left\{ v \in F^{*}: \exists c \geq 0 \hspace{4pt} \vert \langle v, Au \rangle \vert \leq c \Vert u \Vert \hspace{4pt} \forall u \in D(A) \right\}
	\]
	Now we go about defining $A^{*}v$. Given $v \in D(A^{*})$, we define $g(u) = \langle v, Au \rangle$. Use Hahn-Banach to extend $g$ to a bounded functional $f \in E^{*}$, which is unique if $D(A)$is dense in $E$. Let $A^{*}v = f$. In brief, 
	\[
	\langle v, Au \rangle_{F^{*}, F} = \langle A^{*}v, u \rangle_{E^{*},E}
	\]
	
\end{defn}

\begin{prop}
	Let $A: D(A) \subset E \to F$ be a densely defined unbounded linear operator. Then $A^{*}$ is closed. 
\end{prop}

\begin{cor}
	Let $A: D(A) \subset E \to F$ be an unbounded linear operator that is densely defined and closed. Then
	\begin{enumerate}
		\item $N(A) = R(A^{*})^{\perp}$
		\item $N(A^{*}) = R(A)^{\perp}$
		\item $N(A)^{\perp} \supset \overline{R(A^{*})}$
		\item $N(A^{*})^{\perp} = \overline{R(A)}$
	\end{enumerate}
\end{cor}

\subsection{A Characterization of Operators with Closed Range. A Characterization of Surjective Operators}

\begin{thm}
	Let $A: D(A) \subset E \to F$ be an unbounded linear operator that is densely defined and closed. The following are equivalent:
	\begin{enumerate}
		\item $R(A)$ is closed
		\item $R(A^{*})$ is closed
		\item $R(A) = N(A^{*})^{\perp}$
		\item $R(A^{*}) = N(A)^{\perp}$
	\end{enumerate}
\end{thm}

\begin{thm}
	Let $A: D(A) \subset E \to F$ be a linear operator that is densely defined and closed. The following are equivalent:
	\begin{enumerate}
		\item $A$ is surjective
		\item There is a constant $C$ such that
			\[
			\Vert v \Vert \leq C \Vert A^{*}v \Vert
			\]
		\item $N(A^{*}) = \left\{ 0 \right\}$ and $R(A^{*})$ is closed.
	\end{enumerate}
\end{thm}

\begin{thm}
	Let $A: D(A) \subset F$ ($ \to E^{*}$?) be an unbounded linear operator that is densely defined and closed. The following are equivalent:
	\begin{enumerate}
		\item $ A^{*}$ is surjective
		\item there is a constant $C$ such that
			\[
			\Vert u \Vert \leq C \Vert Au \Vert
			\]
		\item $N(A) = 0$ and $R(A)$ is closed.
	\end{enumerate}
\end{thm}

