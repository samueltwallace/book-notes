\section{Topology of Differentiable Manifolds}
\subsection{Complexes}
\begin{defn}

	A \textit{closure finite abstract complex} $K$ is a countable collection of object $ \left\{ S^p_i \right\}$ called \textit{simplexes} satisfying the following properties:
	\begin{enumerate}
		\item To each simplex $S^p_i$ there is associated an integer $p \geq 0$ called its \textit{dimension};
		\item To the simplexes $S^p_i$ and $S^{ p-1 }_j$ is associated an integer denoted by $ \left[ S^p_i : S^{ p-1 }_j \right]$ called their \textit{incidence number};
		\item There are only a finite number of simplexes $ S^{ p-1 }_j$ such that $ \left[ S^p_i : S^{ p-1 }_j \right] \neq 0$;
		\item For every pait of simplexes $S^{ p+1 }_i, S^{ p-1 }_j$ whose dimensions differ by two
			\[
			\sum_k \left[ S^{ p+1 }_i : S^p_k \right] \left[ S^p_k : S^{ p-1 }_j \right] = 0
			\]
			
	\end{enumerate}
	We associate with $K$ an integer $ \mathrm{dim}K$ called its dimension which is the max dimension of its simplexes.

\end{defn}

\begin{defn}

	An algebraic structure is imposed on $K$ as follows: the $p$-simplexes are taken as free generators of an abelian group. A finite sum
	\[
		C_p = \sum_i g_i S^p_i; \hspace{4pt} g_i \in G
	\]
	where $G$ is an abelian group group is called a \textit{$p$-dimensional chain} or a \textit{$p$-chain}. Two $p$-chains may be addded, with their sum being the sum of their coefficients of each simplex. This way, $p$ chains form an abelian group denoted by $ C_p(K,G)$.

\end{defn}

\begin{defn}

	Let $ \Lambda$ be a ring with unity 1. A $ \Lambda$-module is an abelian group $A$ together with a map $ \left( \lambda,a \right) \to \lambda a$ of $ \Lambda \times A \to A$ satisfying
	\begin{enumerate}
		\item $ \lambda ( a_1 + a_2) = \lambda a_1 + \lambda a_2$
		\item $ ( \lambda_1 + \lambda_2) a = \lambda_1 a + \lambda_2 a$
		\item $ ( \lambda_1 \lambda_2) a = \lambda_1 \left( \lambda_2 a \right)$
		\item $1a=a$
	\end{enumerate}
\end{defn}


\begin{defn}

	Let $A$ be a right $ \Lambda$-module and $B$ a left $ \Lambda$-module. Let $F_{ A \times B }$ the free abelian group having as a basis the set $A \times B$ of pairs $(a,b)$ and let $ \Gamma$ be the subgroup of $F_{ A \times B }$ the subgroup of $F_{ A \times B }$ generated by elements of the form
	\[
		(a_1 + a_2,b) - (a_1,b) - (a_2,b)
	\]
	\[
		(a,b_1 + b_2) - (a,b_1) - (a,b_2)
	\]
	\[
		(a \lambda, b) - (a, \lambda b)
	\]
	The quotient group $ F_{ A \times B } / \Gamma$ is called the tensor product of $A$ and $B$ and it is an abelian group.

\end{defn}

\begin{defn}

	The boundary map $ \partial: C_p(K,G) \to C_{ p-1 }(K,G)$ is defined by the formula
	\[
	\partial C_p = \sum_i g_i \partial S^p_i = \sum_i \sum_j g_i \left[ S^p_i : S^{ p-1 }_j \right] S^{ p-1 }_j
	\]
	where since $ \left[ S^p_i : S^{ p-1 }_j \right]$ is an integer, its multiplication against $g_i$ is considered as a multiple of $g_i$ in the $ \mathbb{Z}$-module of $G$. As a linear function, the boundary map is a group homomorphism.

\end{defn}

\begin{defn}

	The kernel of $ \partial$ is denoted by $Z_p(G,K)$, and its elements are called \textit{$p$-cycles}. Since $ \partial^2 = 0$, the set of $p$-cycles contains the image of $\partial$ on $C_{ p-1 }(K,G)$, denoted by $B_p(K,G)$ whose elements are called boundaries. The quotient group
	\[
		H_p(K,G) = Z_p(K,G)/B_p(K,G)
	\]
	is called the \textit{$p$-th homology group} of $K$ with coefficient group $G$. the elements of $H_p(K,G)$ are called \textit{homology classes}.

\end{defn}

\begin{defn}

	Let $C_p(K) = C_p(K, \mathbb{Z})$, elements of which we will call \textit{integral $p$-chains} of $K$. A linear function $f^p$ defined on $C_p(K)$ with values in a commutative topological group $G$:
	\[
		f^p: C_p(K) \to G
	\]
	is called a \textit{p-dimensional cochain} or a \textit{p-cochain}. We define groups dual to the homology groups by using function addition as the group operation on $p$-cochains.

\end{defn}


\begin{defn}

The operator $\partial^*$ dual to $ \partial$ called hte \textit{coboundary operator} is defined by 
\[
	\left( \partial^* f \right)(C_{ p+1 }) = f^p \left( \partial C_{ p+1 } \right)
\]
It is a linear, square-free map.
\end{defn}

\begin{defn}

	The kernel of $\partial^*$ is denoted by $Z^p(K,G)$ and its elements are called \textit{$p$-cocycles}. The image of $C^{ p-1 }(K,G)$ under $ \partial^*$ is denoted by $B^p(K,G)$ and its elements are called \textit{coboundaries}. The quotient group 
	\[
		H^p(K,G) = Z^p(K,G) / B^p(K,G)
	\]
	is called the \textit{$p$-th cohomology group} of $K$ wit coefficient group $G$. Its elements are called \textit{cohomology classes}.

\end{defn}

\subsection{Singular Homology}

\begin{defn}

A \textit{geometric realization} $K_E$ of an abstract complex $K$ we mean a complex whose simplexes are geometric simplexes; i.e., points, lines, triangles, tetrahedrons in Euclidean space $ \mathbb{R}^n$ of sufficiently high dimension, in such a way that distinct abstract simplexes correspond to disjoint geometric simplexes. The union of all the simplexes in $K_E$, written $ \vert K_E \vert$ is called a \textit{polyhedron} and the abstract complex is said to be a \textit{covering} of $ \vert K_E \vert$.

\end{defn}

\begin{defn}

Two complexes are \textit{isomorphic} if there is a bijection between the two preserving incidences.

\end{defn}

\begin{prop}

Isomorphic complexes induce a homeomorphism between their geometric realizations. The homology groups of isomorphic complexes are isomorphic.

\end{prop}

\begin{defn}

	If the group of coefficient $G$ form a ring $F$, the homology groups become modules over $F$. The rank of $H_p(K,F)$ as a module over $F$ is called the \textit{$p$-th betti number} $b_p(K)$. If $F$ has characteristic zero, $H_p(K)$ is a vector space. The expression $\sum_p (-1)^p b_p(K)$ is called the \textit{Euler-Poincar\'e characteristic} of $K$.

\end{defn}

\begin{defn}

A \textit{$p$-simplex} $ \left[ \phi: S^p \right]$ on a differentiable manifold $M$ is a geometric simplex and a differentiable map $ \phi: S^p \to M$. A \textit{singular $p$-chain} $s^p$ on $M$ is a formal sum of $p$-simplexes with coefficients in a group $G$. \\
The support of $s^p$ is $ \phi \left( S^p \right)$, and a chain is \textit{locally finite} if each compact set in $M$ meets only a finite number of supports with $g_i \neq 0$. \\
The \textit{faces} of a $p$-simplex $s^p = \left[ \phi,S^p \right]$ are the simplexes $ \phi \left( S^{ p-1 }_i \right)$. A boundary operator $ \partial$ is defined by putting
\[
	\partial s^p = \sum_i (-1)^i s^{ p-1 }_i
\]
and $ \partial s^0 = 0$. Cycles and boundaries are defined with respect to this boundary map, giving rise to the \textit{$p^{ th }$ singular homology space} of $M$, ddenoted $SH_p$.

\end{defn}

\subsection{Stokes' Theorem}































