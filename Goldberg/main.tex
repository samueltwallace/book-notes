\documentclass{../booknotes}

\usepackage{import}

\booktitle{Curvature and Homology}
\bookauthor{Samuel I. Goldberg}
\notesauthor{Samuel T. Wallace}

\begin{document}

\maketitle

\begin{pubdescrip}
	\indent \indent This systematic and self-contained treatment examines the topology of differentiable manifolds, curvature and homology of Riemannian manifolds, compact Lie groups, complex manifolds, and curvature and homology of Kaehler manifolds. It generalizes the theory of Riemann surfaces to that of Riemannian manifolds. Includes four helpful appendixes
\end{pubdescrip}

\begin{transcribernote}
	\indent These notes were taken for a self-study program to familiarize myself with homology and cohomology theories on different types of manifolds without getting too into the weeds on algebraic topology. This isn't the most modern book, and some of its approaches are outdated. To give an example, I went to a talk on shellability of complexes at my school, and the definition of a simplicial complex was more constructive, but just as abstract, as the definition in the book. This book still provides a valuable introduction to the topic though, and covers material relevant to a modern geometer. \\
	\indent The reader should be very comfortable with different types of manifolds. Riemannian, complex, and K\"ahler manifolds all appear, and the introduction to them is more than brief. A companion book on the different structures on manifolds might be helpful for the unfamiliar reader. Some analysis may be helpful as well, in terms of operators and infinite-dimensional vector spaces. Nothing more than the spectral theorem is strictly required, and I haven't even had to call on that yet.
\end{transcribernote}

\tableofcontents

\import{./}{ch2.tex}
\import{./}{ch3.tex}


\end{document}
