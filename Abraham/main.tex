\documentclass{../booknotes}

\booktitle{Foundations of Mechanics}
\bookauthor{Ralph Abraham and Jerrold E. Marsden}
\notesauthor{Samuel T. Wallace}

\usepackage{import} 

\begin{document}

\maketitle

\begin{pubdescrip}
\indent \indent A reference on symplectic geometry, analytical mechanics and symplectic methods in mathematical physics. It offers a treatment of geometric mechanics. It is also suitable as a textbook for the foundations of differentiable and Hamiltonian dynamics.
\end{pubdescrip}

\begin{transcribernote}
	\indent These notes were taken as part of an independent study class at the University of Florida in Fall 2019. Assuming familiarity with modern differential geometry, these notes delve right into the basics up to infinite-dimensional versions of symplectic geometry and edges into symplectic topology. The main focus of these notes are Hamiltonian dynamics on different symplectic manifolds. Also included are topological approaches to dynamics, such as Smale's notion of a \textit{simple mechanical system}.\\
	\indent These notes are suitable for a late undergrad, early grad student interested in the abstraction of mechanics. It could be worthwhile to mathematicians and physical scientists alike. Physicists with a mathematical bent will probably be interested in these as an introduction to Hamiltonian dynamics in the mathematical sense, to prepare for quantum mechanics, or the ADM formalism.
	\indent These notes were taken without proofs, mainly as a cheat sheet for myself. They are meant as a `theorem cheat sheet', a reference guide for important theorems, rather than as a primary teaching tool. Nevertheless, excepting some aspects of modern differential geometry, this is self-contained.
\end{transcribernote}

\tableofcontents

\import{./}{ch3.tex}
\import{./}{ch4.tex}
\import{./}{ch5.tex}


\end{document}
