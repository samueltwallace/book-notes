\documentclass{article}

\usepackage{amsmath}
\usepackage{amssymb}
\usepackage{mathrsfs}
\usepackage{mathtools}
\newtheorem{thm}{Theorem}
\newtheorem{defn}{Definition}
\newtheorem{prop}{Proposition}
\newtheorem{rmk}{Remark}
\newtheorem{cor}{Corollary}
\newtheorem{lem}{Lemma}




\begin{document}

\section{Linear Sympletic Geometry}
\subsection{Symplectic Vector Spaces}
\begin{defn}

A \textbf{Symplectic Vector Space} is a pair $(V,\omega)$ of a finite-dimensional vector space $V$ and a non-degenerate skew-symmetric bilinear form $\omega: V \times V \to \mathbb{R}$. A \textbf{Linear Symplectomorphism} is a linear map preserving the symplectic form. The set of linear symplectomorphisms is denoted by $\mathrm{Sp}(V,\omega)$.

\end{defn}

\begin{defn}

The symplectic complement of a subspace $W$ is the subspace $W^{\omega} = \{v \in V \vert \omega(v,w)=0 \hspace{4pt} \forall w \in W\}$. Subspaces are called \\

\begin{tabular}{l|l}
     isotropic & $W \subset W^{\omega}$ \\
     coisotropic & $W^{\omega} \subset W$ \\
     symplectic & $W \cap W^{\omega} = \{0\}$ \\
     Lagrangian & $W = W^{\omega}$
\end{tabular}

\end{defn}

\begin{lem}

$\mathrm{dim}W + \mathrm{dim}W^{\omega} = \mathrm{dim}V$; $W^{\omega \omega} = W$

\end{lem}

\begin{cor}

$\omega$ is nondegenerate iff the associated volume element is nonzero: $\omega \wedge \ldots \wedge \omega = \omega^n$

\end{cor}

\begin{lem}

Any isotropic subspace is contained in a Lagrangian subspace. Additionally, any basis of a Lagrangian subspace can be extendded to a sympletic basis of $(V,\omega)$.

\end{lem}

\begin{lem}

Let $(V,\omega)$ be a symplectic vector space of $W \subset V$ a coisotropic subspace. Then: 
\begin{enumerate}
    \item $V'=W/W^{\omega}$ carries a natural symplectic structure $\omega '$ induced by $\omega$.
    \item If $\Lambda \subset V$ is a Lagrangian subspace then $\Lambda ' = ((\Lambda \cap W) + W^{\omega})/W^{\omega}$ is a Lagrangian subspace of $V'$
    
\end{enumerate}

\end{lem}

\subsection{The Symplectic Linear Group}

\begin{lem}

$\mathrm{Sp}(2n) \cap O(2n) = \mathrm{Sp}(2n) \cap \mathrm{GL}(n,\mathbb{C}) = \mathrm{U}(n)$

\end{lem}

\begin{lem}

$\lambda \in \sigma(\Psi) \iff \lambda^{-1} \in \sigma(\Psi)$, and their multiplicities are identical (Here $\Psi$ is a linear symplectomorphism, and $\sigma(\Psi)$ its spectrum). Moreover, distinct eigenvectors $z_1, z_2$ have the property that $\omega(z_1, z_2) = 0$. 

\end{lem}

\begin{lem}

Every real symmetric positive-definite symplectic matrix, taken to any positive real power, is again a symplectic matrix

\end{lem}

\begin{prop}

$\mathrm{U}(n)$ is a maximal compact subgroup of $\mathrm{Sp}(2n)$ and the quotient $\mathrm{Sp}(2n)/\mathrm{U}(n)$ is contractible. 

\end{prop}

\begin{prop}


The fundamental group of $\mathrm{U}(n)$ is isomorphic to the integers. The determinant map $\mathrm{det}: \mathrm{U}(n) \to \mathrm{S}^1$ induces an isomorphism of fundamental groups.

\end{prop}

\subsection{The Maslov Index}

\begin{thm}

There is a unique function $\mu: \Omega \mathrm{Sp}(2n) \to \mathbb{Z}$ satisfying the following: 
\begin{enumerate}
    \item (homotopy) Two loops have the same Maslov index $\iff$ they are homotopic
    \item (product) For any two loops $\Psi_t, \Phi_t: \mathbb{R/Z} \to \mathrm{Sp}(2n)$, $\mu(\Psi_t \Phi_t) = \mu(\Psi_t) + \mu(\Phi_t)$
    \item (direct sum) Identifying $ \mathrm{Sp}(2a) \bigoplus \mathrm{Sp}(2b) \subset \mathrm{Sp}(2a+2b)$, then $\mu(\Psi \bigoplus \Phi) = \mu(\Psi) + \mu (\Phi) $
    \item (normalization) The loop $\theta_t = e^{2 \pi i t} \in \mathrm{U}(1)$ has Maslov index 1.
\end{enumerate}

\end{thm}

The Maslov index can also be considered the intersection number of a loop with a certain submanifold. Decompose a symplectic matrix into a 2x2 block matrix form, then take the upper right matrix and set its determinant equal to zero. This forms a codimension one submanifold. Then take the Maslov index to be the intersection number of the loop with this submanifold.

\subsection{Lagrangian Subspaces}

Let $\mathcal{L}(V,\omega)$ be the set of Lagrangian subspaces of $(V,\omega)$

\begin{lem}

Let $X$ and $Y$ be real $n \times n$ matrices and define $\Lambda \subset \mathbb{R}^{2n} $ by $\Lambda = \mathrm{range}(Z)$; $Z = (X Y)^{T}$. Then $\Lambda$ is a Lagrangian subspace $\iff$ $Z$ is of full rank and $X^T Y = Y^T X$

\end{lem}

The matrix that satisfies the above is a \textbf{Lagrangian Frame}. $\mathcal{L}(n)$ is a manifold of dimension $n(n+1)/2$.

\begin{lem}

\begin{enumerate}
    \item Any symplectic transformation of a Lagrangian subspace is again Lagrangian
    \item There is a symplectic transform between any two Lagrangian subspaces.
    \item There is a natural isomorphism $\mathcal{L}(n) \approx \mathcal{U}(n)/\mathcal{O}(n)$.
\end{enumerate}

\end{lem}

\begin{thm}

There is a unique function $\mu: \Omega \mathcal{L}(n) \to \mathbb{Z}$ satisfying the following:

\begin{enumerate}
    \item (homotopy) Two loops have the same Maslov index $\iff$ they are homotopic
    \item (product) For a loop $\Lambda_t \in \Omega \mathcal{L}(n)$ and a loop $\Psi_t \in \Omega \mathrm{Sp}(2n)$, then $\mu(\Psi_t \Lambda_t) = \mu(\Lambda_t) + 2 \mu(\Psi_t)$
    \item (direct sum) Identifying $\mathcal{L}(a) \bigoplus \mathcal{L}(b) \subset \mathcal{L}(a+b)$, then the Maslov index of a direct sum of two loops is the sum of their Maslov indices.
    \item (normalization)  The loop $\Lambda_t = e^{2 \pi i t} \mathbb{R} \subset \mathbb{C}$ has Maslov index 1.
    
\end{enumerate}

\end{thm}

Similarly to the Maslov indices of symplectomorphism loops, we can view the Maslov index of Lagrangian subspaces as the intersection number of the loop with a submanifold of $\mathcal{L}(n)$. The desired submanifold is the set of planes $\bigcup_{c \in \mathbb{R}} \{\mathrm{Re}(z) = c\}$ in complex symplectic basis.


\subsection{The Affine Non-Squeezing Theorem}

\begin{defn}

An \textbf{Affine Symplectomorphism} is a map that is a symplectomorphism followed by a translation. A symplectic cylinder $Z^{2n}(R)$ is $B^2 (R) \times \mathbb{R}^{2n-2}$

\end{defn}

\begin{thm}

Let $\psi$ be an affine symplectormophism, and that $\psi(B^{2n}(r)) \subset Z^{2n}(R)$. Then $r \leq R$.

\end{thm}

\begin{thm}

Let $\Psi$ be a nonsingular matrix with the non-squeezing property. Then $\Psi$ is symplectic or anti-symplectic ($\Psi^*\omega = -\omega)$. 

\end{thm}

\begin{defn}

The \textbf{Linear Symplectic Width} of a subset $A \subset \mathbb{R}^{2n}$ is the area of the largest symplectic ball that fits inside the subset: 
\[
w_L(A) = \sup \{\pi r^2 \vert \psi(B^{2n}(r)) \subset A\}
\]
For any affine symplectic transform $\psi$. It has the following properties:

\begin{enumerate}
    \item (monotonicity) if $\psi(A) \subset B$ then $w_L(A) \leq w_L(B)$.
    \item (conformality) $w_L(\lambda A) = \lambda^2 w_L(A)$
    \item (nontriviality) $w_L(B^{2n}(r)) = w_L(Z^{2n}(r)) = \pi r^2$
\end{enumerate}
\end{defn}

\begin{thm}

Let $\Psi: \mathbb{R}^{2n} \to \mathbb{R}^{2n}$ be a linear map. Then TFAE:

\begin{enumerate}
    \item $\Psi$ preserves the linear symplectic width of ellipsoids centered at $0$.
    \item $\Psi$ is either symplectic or anti-symplectic, i.e. $\Psi^* \omega = \pm \omega$
\end{enumerate}

\end{thm}

\begin{lem}

Let $(V,\omega)$ be a symplectic vector space with an inner product $g$. Then there is a basis of $V$ which is $g$-orthogonal and $\omega$-standard, and can be chosen so that $g(u_j, u_j) = g(v_j, v_j)$

\end{lem}

\begin{lem}

Given an ellipsoid
\[E = \{W \in \mathbb{R}^{2n} \vert \sum_{i,j=1}^{2n} a_{ij} w_i w_j \leq 1 \}\]

There is a symplectic linear tranformation $\Psi in \mathrm{Sp}(2n)$ such that $\Psi$ turns the matrix $a_{ij}$ diagonal.

\end{lem}

\begin{rmk}

For $n=1$, the previous lemma tells us that every ellipse in $\mathbb{R}^2$ can be mapped into a circle by an area-preserving transformation. 

\end{rmk}

\begin{defn}

The \textbf{Symplectic Spectrum} of an ellipsoid to be the increasing $n$-tuple $(r_1, \ldots, r_n)$ such that the ellipsoid is 'diagonalized' by a linear transformation into a ellipsoid with diagonal matrix $\mathrm{diag}(r_1, \ldots, r_n)$. Symplectic spectra have the following properties:

\begin{enumerate}
    \item Two ellipsoids are linearly symplectomorphic $\iff$ they have the same spectrum
    \item An ellipsoid with its spectrum of the form $(r,\ldots,r)$ are symplectic balls
    \item The volume of a symplectic ellipsoid is $\mathrm{Vol}(E) = \pi^n \Pi_i r_i^2$
\end{enumerate}

\end{defn}


\begin{thm}

Let $E \subset \mathbb{R}^{2n}$ be an ellipsoid centered at 0. Then $ w_L(E) = \sup_{B\subset E} w_L(B) = \inf_{E \subset Z} w_L(Z)$, where $B$ are symplectic balls and $Z$ are symplectic cylinders.

\end{thm}

\subsection{Complex Structures}

\begin{defn}

A \textbf{Complex Structure} is an automorphism $J: V \to V$ such that $-J^2$ is the identity. We can 'complexify' the vector space by ''complex' scalar multiplication: $\mathbb{C}\times V \to V: (s+i t, v) \mapsto sv + tJv$, which means $V$ has even dimension over the reals. The set of complex structures is denoted by $\mathcal{J}(V)$

\end{defn}

\begin{prop}

Every almost complex structure is a linear transform away from the standard complex structure:

\[
J = \begin{bmatrix}
     0 & -\mathrm{id} \\
     \mathrm{id} & 0 \\
\end{bmatrix}
\]

\end{prop}

\begin{defn}

If $(V,\omega)$ is symplectic vector space, a complex structure $J \in \mathcal{J}(V)$ is said to be compatible with $\omega$ if $\omega(Jv, Jw) = \omega(v,w)$ and $\omega(v,Jv) > 0$. This defines an inner product on $V$, defined by $g_J(v,w) = \omega(w,Jw)$. This inner product makes $J$ skew-adjoint, i.e. $g_J(v,Jw) = - g_J(Jv, w)$. The space of compatible complex structures is denotes $\mathcal{J}(V,\omega)$
\end{defn}

\begin{prop}

\begin{enumerate}
    \item $\mathcal{J}(V,\omega)$ is homeomorphic to the space $\mathcal{P}$ of symmetric positive definite symplectic matrices.
    \item There is a continuous map $r:\text{met}(V) \to \mathcal{J}(V,\omega)$ such that $r(g_J) = J$ \& $r(\Phi^* g) = \Phi^* r(g)$, $\forall J \in \mathcal{J}(V,\omega), g \in \text{met}(V), \Phi \in \text{Sp}(V,\omega)$.
    \item $\mathcal{J}(V,\omega)$ is contractible.
    
    
\end{enumerate}

\end{prop}

\begin{defn}

A complex structure $J \in \mathcal{J}(V,\omega)$ is called $\omega$-tame if $\omega(v,Jv) > 0$ for all nonzero $v$. The space of $\omega$-tame complex structures is denotes by $\mathcal{J}_T(V, \omega)$. There is an associated inner product for each $\omega$-tame complex structure given by $g_J(v,w) = \frac{1}{2}(w(v,Jw) + \omega(w,Jv))$.

\end{defn}

\begin{prop}

The space $\mathcal{J}_T(V,\omega)$ is contractible.

\end{prop}

\subsection{Symplectic Vector Bundles}

\begin{defn}

A \textbf{Symplectic Vector Bundle} over a manifold is a real vector bundle $E \overset{\pi}{\to} M$ equipped with a smoothly symplectic form $\omega \in \Gamma(E \bigotimes E^*)$. 

\end{defn}

\begin{defn}

Symplectic vector bundles are isomorphic $\iff$ their underlying complex vector bundles are isomorphic. Symplectic vector bundles with compatible almost complex structure and a metric are called a \textbf{Hermitian Structure}.

\end{defn}

\begin{prop}

Let $E \to M$ be a $2n$-dimensional vector bundle.
\begin{enumerate}
    \item Every symplectic form has a compatible almost complex structure. The space $\mathcal{J}(E,\omega)$ is contractible.
    \item The space of symplectic forms compatible with a given almost complex structure is contractible.
\end{enumerate}

\end{prop}

\begin{defn}

A \textbf{Unitary Trivialization} is a smooth map of a symplectic vector bundle, an almost complex structure, and a metric into Euclidean space, transforming each structure into its standard structure. A \textbf{Unitary Trivialization along a curve} is a trivialization of the pull-back bundle along a curve.
\end{defn}

\begin{lem}

If a curve has unitary trivializations at its endpoints, then it can be extended to a unitary transformation along the entire curve

\end{lem}

\begin{prop}

A Hermitian vector bundle $E \to \Sigma$ over a compact Riemann surface $\Sigma$ with non-empty boundary $\partial \Sigma$ admits a unitary trivialization.

\end{prop}

\subsection{First Chern Classes}

\begin{thm}

There is a unique function called the \textbf{First Chern Number}, that assigns an integer $c_1(E)$ to every symplectic vector bundle $E$ over a compact oriented Riemann surface $\Sigma$ without boundary and satisfies the following axioms:

\begin{enumerate}
    \item (naturality) Two isomorphic vector bundles have the same Chern number
    \item (functoriality) Any smooth map $\phi: \Sigma' \to \Sigma$ of oriented Riemann surfaces and any symplectic vector bundle $E \to \Sigma$, then $c_1(\phi^*E) = \mathrm{deg}(\phi) c_1(E)$
    \item (additivity) For any two symplectic vector bundles $E_1 \to \Sigma$ and $E_2 \to \Sigma$, $c_1(E_1 \bigoplus E_2) = c_1(E_1) + c_1(E_2)$.
    \item (normalization) The Chern number of the tangent bundle is $c_1(T\Sigma) = 2 - 2g = \chi(\Sigma)$
\end{enumerate}

\end{thm}

\begin{rmk} 

\begin{enumerate}
    \item The first Chern number vanishes $\iff$ the bundle is trivial; so the first Chern number is an indicator of if the bundle can be symplectically trivialized.
    \item Usually the Chern number is defined for complex vector bundles, which is fine for our definition.
\end{enumerate}
\end{rmk}





\end{document}