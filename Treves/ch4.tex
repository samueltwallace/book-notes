\section{Hausdorff Topological Vector Spaces. Quotient Topological Vector Spaces. Continuous Linear Mappings}

Throughout we denote by $ E $ a TVS over the field of complex numbers.

\subsection{Hausdorff Topological Vector Spaces}

\indent A topological space $ X $ is said to be \textit{Hausdorff} if, given any two distinct points $ x $ and $ y $ , there is a neighborhood $ U $ of $ x $ and a neighborhood $ V $ of $ y $ that do not intersect.

\begin{thm}
	A filter on a Hausdorff topological space converges to at most one point.
\end{thm}

\begin{cor}
	Every one-point set in a topological space is closed.
\end{cor}

\begin{prop}
	$ E $ is Hausdorff iff for every point $ x \neq 0 $ there is a neighborhood $ U $ of 0 such that $ x \notin U $ .
\end{prop}

\begin{prop}
	The intersection of all neighborhoods of the origin is a vector space of $ E $ , which is the closure of the origin.
\end{prop}

\begin{cor}
	For $ E $ to be Hausdorff, it is necessary and sufficient that the singleton set containing the origin be closed in $ E $ .
\end{cor}

\begin{prop}
	Let $ f,g $ be two continuous mappings of a topological space $ X $ into $ E $ . The set $ A $ on which $ f $ and $ g $ are equal in value is closed in $ X $ .
\end{prop}

\begin{prop}
	Let $ X, f, g $ be as in the previous. If $ f $ and $ g $ are equal on a dense subset $ Y $ of $ X $ , then they are equal everywhere on $ X $ .
\end{prop}

\subsection{Quotient Topological Vector Spaces}

Let $ E $ be a TVS and $ M $ a subspace of $ E $ . Then the map $ \phi: E \to E/M $ can be made a quotient map in the topological sense by saying a set $ \dot{U} $ of $ E/M $ if there is an open set $ U \subset E $ such that $ \phi(U) = \dot{U} $ . In other words, we impose the finest topology such that $ \phi $ is an open map (in the topological space). Clearly the quotient topology is compatible with the linear structure on $ E/M $ .

\begin{prop}
	Let $ E $ be a TVS and $ M $ a vector subspace of $ E $ . The following two properties are equivalent:
	\begin{enumerate}
		\item $ M $ is closed
		\item $ E/M $ is Hausdorff
	\end{enumerate}
	
\end{prop}

\begin{cor}
	The TVS $ E/ \overline{ \{ 0 \} } $ is Hausdorff.
\end{cor}


\subsection{Continuous Linear Mappings}
\indent Let $ E $ , $ F $ be two TVS, and $ f: E \to F $ a linear map. If $ F $ is Hausdorff and $ f $ is continuous, then $ \mathrm{ Ker }f $ is is closed. Note that these are not necessary conditions, there are other ways of having a closed kernel.


