\section{Filters. Topological Spaces. Continuous Mappings}

\begin{defn}
	A \textbf{filter} $ \mathcal{F} $  is a family of non-empty subsets of a set $ E $ that meet the following conditions:
	\begin{enumerate}
		\item The intersection of any two sets in $ \mathcal{F} $  is again in $ \mathcal{F} $ .
		\item Any set with a subset in $ \mathcal{F} $ is also in $ \mathcal{F} $ .
	\end{enumerate}

	
\end{defn}


We can generate a filter based on infinite sequences in the set. Let $ ( x_{ n } ) $ be a sequence in $ E $. The associated filter is the family of all subsets of $ E $ that contain all but a finite number of the points of the sequence.

\begin{defn}
	A family $ \mathcal{B} $ of subsets of $ E $ is a \textit{basis} of a filter $ \mathcal{F} $ on $ E $ if the following two conditions are satisfied:

	\begin{enumerate}
		\item $ \mathcal{B} \subset \mathcal{F} $ 
		\item Every element of $ \mathcal{F} $ contains some subset of $ E $ which belongs to $ \mathcal{B} $ .
	\end{enumerate}

\end{defn}

If $ \mathcal{A} $ is a family of subsets of $ E $ , we can generate a filter with $ \mathcal{A} $ as a basis as the set of subsets of $ E $ containing sets of $ \mathcal{A} $ , when we have the requirement that the intersection of any two sets in $ \mathcal{A} $ contain a set which is again in $ \mathcal{A} $ . \\

Filters have a natural ordering induced by set inclusion. We say a filter that contains another filter is finer than its subset (i.e. if $ \mathcal{F} \subset \mathcal{F}' $ , then $ \mathcal{F}' $ is finer than $ \mathcal{F} $ ). \\

\begin{defn}
	A \textit{topology} on a set $ E $ is an assignment to each point $ x \in E $ of a filter $ \mathcal{F} (x) $ on $ E $ with the requirements
	\begin{enumerate}
		\item If a set belongs to $ \mathcal{F} (x) $ then it contains $ x $ .
		\item If a set $ U $ belongs to $ \mathcal{F} (x) $ , there is another set $ V $ belonging also to $ \mathcal{F} (x) $  such that, given any point $ y $ of $ V $ , $ U $ belongs to $ \mathcal{F} (y) $ .
	\end{enumerate}
	We say that we have a topology on $ E $ and we call $ \mathcal{F} (x) $ the \textit{filter of neighborhoods} of the point $ x $ . A basis of the filter $ \mathcal{F} (x)  $ is called a \textit{basis of neighborhoods} of $ x $ .
\end{defn}

\begin{defn}
	The following are defined in fiter-theoretic topology:
	\begin{enumerate}
		\item A set is \textit{open} if it is in the filter of neighborhoods for each of its elements
		\item A closed set is the complement of an open set
		\item The closure of a set is the smallest closed set containing the given set
		\item The interior of a set is the largest open set contained in the given set
		\item A set is \textit{dense} in another set if its closure contains the set.
	\end{enumerate}
	
\end{defn}


