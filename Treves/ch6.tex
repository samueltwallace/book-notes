\section{Compact Sets}

A topological space $ X $ is said to be \textit{compact} if $ X $ is Hausdorff and if every open cover contains a finite subcovering. 

\begin{prop}
	A closed subset of a compact space is compact.
\end{prop}

\begin{prop}
	Let $ f $ be a continuous mapping of a compact set $ X $ into a Hausdorff topological space $ Y $ . Then $ f(X) $ is a compact subset of $ Y $ .
\end{prop}

\begin{prop}
	Let $ f $ be a 1-1 continuous mapping of a compact space $ X $ onto a compact space $ Y $ . Then $ f $ is a homeomorphism.
\end{prop}

Let $ E $ be a TVS.

\begin{defn}
	$ x \in E $ is an accumulation point of a sequence if it belongs to the closure of the set of points of the sequence.
\end{defn}

\begin{prop}
	If a sequence converges to $ x $ , then $ x $ is an accumulation point of the sequence.
\end{prop}

\begin{prop}
	If a Cauchy sequence in $ E $ has an accumulation point, then it it converges to that point.
\end{prop}

\begin{prop}
	Let $ K $ be a Hausdorff topological space. The following are equivalent:
	\begin{enumerate}
		\item $ K $  is compact
		\item every sequence on $ K $ has at least one accumulation point
	\end{enumerate}
\end{prop}

\begin{cor}
	A compact subset $ K $ of a Hausdorff topological space $ E $ is closed.
\end{cor}

\begin{cor}
	In compact topological spaces, every sequence has an accumulation point.
\end{cor}

\begin{defn}
	$ A \subset X $ is said to be relatively compact (or precompact) if the closure of $ A $  is compact.
\end{defn}

