\section{Locally Convex Spaces. Seminorms}

A subset $ K $ of a vector space $ E $ is convex if, whenever $ K $ contains two points $ x $ and $ y $ , $ K $ also contains the straight line joining them. Now let $ S $ be any subset of $ E $ . We define the \textit{convex hull} of $ S $ to be the set of all finite linear combinations with nonnegative coefficients that sum to one. A set is convex if it is equal to its own convex hull. \\
\indent Intersections of convex subsets, but unions may not be. Sums of convex sets are convex, and the image of convex sets under linear maps are convex. \\

\begin{prop}
	Let $ E $ be a TVS. The closure and interior of convex sets are convex.
\end{prop}

\begin{defn}
	A subset $ T $ of a TVS $ E $ is called a barrel if $ T $ has the following four properties:
	\begin{enumerate}
		\item $ T $ is absorbing
		\item $ T $ is balanced
		\item $ T $ is closed
		\item $ T $ is convex
	\end{enumerate}
	We can construct barrels in the following way: let $ U $ be a neighborhood of 0 in $ E $ . Then we define $ T(U) $ the closure of the convex hull of the set 
	\[
	\bigcup_{ \vert \vert \lambda \vert \leq 1 \vert } \lambda U
	\]
	And $ T(U) $ is a barrel.
\end{defn}

\begin{defn}
	A TVS $ E $ is said to be a locally convex space if there is a neighborhoods in $ E $ consisting of convex sets.
\end{defn}

\begin{prop}
	In a locally convex space $ E $ , there is a basis of neighborhoods of zero consisting of barrels.
\end{prop}

\begin{defn}
	A nonnegative function $ p: E \to \mathbb{R} $ on a vector space $ E $ is called a seminorm if it satisfies the following conditions:
	\begin{enumerate}
		\item $ p $ is subadditive: $ p(x+y) \leq p(x) + p(y) $ 
		\item $ p $ is positively homogeneous of degree 1: $ p( \lambda x) = \vert \lambda \vert p(x) $ 
		\item $ p(0) = 0 $ 
	\end{enumerate}
	A seminorm on a vector space $ E $ is called a norm if $ p(x) = 0 \Rightarrow x=0 $ .
\end{defn}

\begin{defn}
	A vector space $ E $ over the field of complex numbers, provided with a Hermitian nonnegative form,  is called a complex pre-Hilbert space.
\end{defn}

\begin{defn}
	Let $ E $ be a vector space, and $ p $ a seminorm on $ E $ . The sets
	\[
		U_{ p } = \{ x \in E : p(x) \leq 1 \} \hspace{4pt} \dot{U}_{ p } = \{ x \in E: p(x) < 1 \}
	\]
	are called the closed and open unit semiball of $ p $ .
\end{defn}

\begin{prop}
	Let $ E $ be a topological vector space, and $ p $ a seminorm on $ E $ . Then the following are equivalent:
	\begin{enumerate}
		\item the open unit semiball of $ p $ is an open set
		\item $ p $ is continuous at the origin
		\item $ p $ is continuous at every point
	\end{enumerate}
	
\end{prop}

\begin{prop}
	If $ p $ is a continuous seminorm on a TVS $ E $ , its closed unit semiball is a barrel.
\end{prop}

\begin{prop}
	The $ E $ be a TVS, and $ T $ a barrel in $ E $ . There exists a unique seminorm $ p $ on $ E $ such that $ T $ is the closed unit semiball of $ p $ . The seminorm $ p $ is continuous if and only if $ T $ is a neighborhood of $ 0 $ .
\end{prop}

\begin{cor}
	Let $ E $ be a locally convex space. The closed unit semiballs of the continuous seminorms on $ E $ form a basis of neighborhoods of the origin.
\end{cor}

\begin{defn}
	A family $ \mathcal{P} $  of continuous seminorms on a locally convex space $ E $ will be called a basis of continuous seminorms on $ E $ if for any continuous seminorm $ p $ on $ E $ there is a continuous seminorm $ q $ belonging to $ \mathcal{P} $ and a constant $ C > 0 $ such that $ p(x) \leq C q(x) $ .
\end{defn}

\begin{prop}
	Let $ \mathcal{P} $ be a basis of continuous seminorms on the locally convex space $ E $ . Then the sets $ \lambda U_{ p } $ where $ U_{ p } $ is the closed unit semiball of $ p  \in \mathcal{P} $ and $ \lambda $ is a positive number, form a basis of neighborhoods of 0. Conversely, given a family, $ \mathcal{B} $  of neighborhoods of zero and consisting of barrels and such that the sets $ \lambda U, \hspace{4pt} U \in \mathcal{B} $ form a basis of neighborhoods of 0 in $ E $ , then the seminorms whose closed unit semiballs are the barrels belonging to $ \mathcal{B} $ form a basis of continuous seminorms in $ E $ .
\end{prop}

We shall even say that a basis of continuous seminorms on a locally convex space $ E $ \textit{defines} the topology of $ E $ . We shall also use the expression \textit{"a family of seminorms on E defining the topology of E"} in which the family under consideration need not be a basis of continuous seminorms. The meaning it is the following: first: every seminorm $ p_{ \alpha } $ is continuous; second, the family obtained by forming the supremum of finite numbers of seminorms $ p_{ \alpha } $ is a basis of of continuous seminorms on $ E $ . This family consists of the seminorms
\[
	x \mapsto p_{ (B) } (x) = \sup_{ \alpha \in B } p_{ \alpha } (x)
\]
where $ B $ ranges over all the finite subsets of the set of indices $ A $ of the family $ \{ p_{ \alpha } \} $ . Forming the supremum of a finite number of seminorms is the equivalent of forming the intersection of their closed unit semiballs and taking the "gauge" of this intersection (a seminorm $ p $ is the \textit{gauge} of a set $ U $ if $ U $ is the closed unit semiball of $ p $ ).

\begin{prop}
	Let $ E,F $ be two locally convex spaces. A linear map $ f:E \to F $ is continuous if and only if to every continuous seminorm $ q $ on $ F $ there is a continuous seminorm $ p $ on $ E $ such that
	\[
		q \left( f\left( x \right) \right) \leq p(x)
	\]
\end{prop}

\begin{cor}
	A linear form $ f $ on a locally convex space, $ E $ , is continuous if and only if there is a continuous seminorm $ p $ on $ E $ such that $ \vert f(x) \vert \leq p(x) $ .
\end{cor}

\begin{prop}
	Let $ E $ be a locally convex space, and $ M $ a linear subspace of $ E $ . Let $ \phi $ be the canonical mapping of $ E \to E/M $ . Then the following are true:
	\begin{enumerate}
		\item the topology of the quotient $ TVS $ is locally convex
		\item if $ \mathcal{P} $ is a basis of continuous seminorms on $ E $ , let us dentoe by $ \dot{ \mathcal{P} } $ the family of seminorms on $ E/M $  consisting of the seminorms
			\[
				\dot{x} \mapsto \dot{p}(\dot{x}) = \inf_{ \phi(x) = \dot{x} } p(x)
			\]
		Then $ \dot{ \mathcal{P} } $ is a basis of continuous seminorms on  $ E/M $.
	\end{enumerate}
\end{prop}

We call the \textit{kernel} of a seminorm $ p $ on $ E $ the set of vectors with vanishing seminorm. This is a subspace of $ E $ and we denote is by $ \mathrm{ Ker }p $ . It is closed, and in a locally convex space $ E $ , the closure of the origin is 
\[
\bigcap_{ p } \mathrm{ Ker } p
\]

\begin{prop}
	In a locally convex space $ E $ , the closure of the singleton set containing zero is the intersection of the closed linear subspaces $ \mathrm{ Ker }p $ , where $ p $ varies over a basis of continuous seminorms on $ E $ .
\end{prop}

\begin{prop}
	Let $ E $ be a locally convex Hausdorff TVS, and $ K $ a precompact subset of $ E $ . Then the convex hull $ \Gamma(K) $ of $ K $ is precompact.
\end{prop}

\begin{cor}
	If $ E $ is complete the closed convex hull of a compact subset of $ E $ is compact.
\end{cor}

