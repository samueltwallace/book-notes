\documentclass{../booknotes}

\usepackage{import}

\booktitle{Lectures on Riemann Surfaces}
\bookauthor{Otto Forster}
\notesauthor{Samuel T. Wallace}

\begin{document}

\maketitle

\begin{pubdescrip}
This book grew out of lectures on Riemann surfaces given by Otto Forster at the universities of Munich, Regensburg, and Münster. It provides a concise modern introduction to this rewarding subject, as well as presenting methods used in the study of complex manifolds in the special case of complex dimension one.
\end{pubdescrip}

\begin{transcribernote}
	\indent These notes were taken as part of self-study. I was interested in covering spaces of open subsets of $ \mathbb{C} $ for exploring two-dimensional potential fluid flow. Though the project has been largely unsuccessful, it was a good learning opportunity. Because I no longer have any project relating specifically to Riemann surfaces, I will likely not continue these notes.\\
	\indent These notes assume very little. One-dimensional complex analysis and some point-set topology (or familiarity with manifolds). These should suit mid-to-late undergrads, and applications outlined should be familiar to them. Graduate students interested in Riemann surfaces should probably seek out a book with a wider scope. For an algebro-geometric perspective of complex geometry, check out Griffiths' and Harris' \textit{Principles of Algebraic Geometry}, which I have taken some notes of.
\end{transcribernote}

\tableofcontents

\import{./}{ch1.tex}


\end{document}
