\documentclass{../booknotes}

\booktitle{Introduction to Dynamical Systems}
\bookauthor{Michael Brin and Garret Stuck}
\notesauthor{Samuel T. Wallace}
\usepackage{import}

\begin{document}

\maketitle

\begin{pubdescrip}
	\indent \indent This book provides a broad introduction to the subject of dynamical systems, suitable for a one- or two-semester graduate course. In the first chapter, the authors introduce over a dozen examples, and then use these examples throughout the book tomotivate and clarify the development of the theory. Topics include topological dynamics, symbolic dynamics, ergodic theory, hyperbolic dynamics, one-dimensional dynamics, complex dynamics, and measure-theoretic entropy. The authors top off hte presentation with some beautiful and remarkable applications of dynamical systesm sto such areas as number theory, data storage, and Internet search engines.\\
\indent \indent The book grew out of lecture ntoes from the graduate dynamical systems course at the University of Maryland, College Park, and reflects not only the tastes of the authors, but also to some extent the collectiv eopinion of hte Dynamics Group at the University of Maryland, which includes experts in virtually every major area of dynamical systems.

\end{pubdescrip}

\begin{transcribernote}
	\indent These notes were taken as part of a independent study course in Fall of 2019. They cover one chapter of the book on hyperbolic dynamics, where differentiable maps on manifolds are iterated and an area of the manifold is stretched and compressed in complementary directions. This did not catch my interest that much, so ther will likely not be more notes from this book. \\
	\indent The reader needs to have a familiarity with differential topology, \`a la Hirsch, \textit{Differential Topology}. Properties of maps between manifolds, transversality, and basic Riemannian geometry is needed, though you can pick it up quickly if you need to through the authors' exposition. 
\end{transcribernote}

\tableofcontents

\import{./}{ch5.tex}

\end{document}

