\documentclass{../booknotes}

\usepackage{import}

\booktitle{Principles of Algebraic Geometry}
\bookauthor{Phillip Griffiths and Joseph Harris}
\notesauthor{Samuel T. Wallace}
\newcommand{\dbar}{\overline{\partial}}

\begin{document}

\maketitle

\begin{pubdescrip}
\indent \indent A comprehensive, self-contained treatment presenting general results of the theory. Establishes a geometric intuition and a working facility with specific geometric practices. Emphasizes applications through the study of interesting examples and the development of computational tools. Coverage ranges from analytic to geometric. Treats basic techniques and results of complex manifold theory, focusing on results applicable to projective varieties, and includes discussion of the theory of Riemann surfaces and algebraic curves, algebraic surfaces and the quadric line complex as well as special topics in complex manifolds.
\end{pubdescrip}

\begin{transcribernote}
	\indent  I took these notes almost on a whim. I wanted to learn more about Riemann surfaces and complex manifolds, and a professor recommended this book to me. I initially skimmed the section on complex manifolds, but came back to revisit the foundational material. I ended up falling in love with complex geometry, and its intricacies.\\
	\indent This book requires a lot, honestly more than I feel comfortable with saying that I know thoroughly. I ended up picking up \textit{Curvature and Homology} to help me with some of the algebraic topology that I hadn't seen before. That said, it claims, and does a reasonable job, of being `self-contained.' The reader, if they really want to deeply understand this book, should look through the contents of Ch. 0 and learn it somewhere else. The introduction is too brief to teach yourself thoroughly, and if you want to make it past Ch. 0, you have to know all of that well. This is a big task to undertake, but it seems to me to be worth it. This is a big book; you should expect to take a while to understand it.
\end{transcribernote}

\tableofcontents

\import{./}{ch0.tex}


\end{document}
